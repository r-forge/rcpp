\documentclass[11pt]{article}
\usepackage{url}
\usepackage{vmargin}
\setpapersize{USletter}
\setmarginsrb{1in}{1in}{1in}{1in}{0pt}{0mm}{0pt}{0mm}
\usepackage{charter}

\newcommand{\proglang}[1]{\textsf{#1}}
\newcommand{\pkg}[1]{{\fontseries{b}\selectfont #1}}

\author{Dirk Eddelbuettel\\ {\small \url{edd@debian.org} }
  \and Romain Fran\c{c}ois\\ {\small \url{romain@r-enthusiasts.com} } }

\title{\pkg{Rcpp}: Seamless \proglang{R} and \proglang{C++} integration}
\date{Submitted to \textsl{useR! 2010}}

\begin{document}

\maketitle
\thispagestyle{empty}
\begin{abstract}
  \addtolength{\parskip}{\baselineskip} 	% add a little vertical space
  \noindent % no ident for first paragraph
  The \pkg{Rcpp} package simplifies integrating \proglang{C++} code with
  \proglang{R}. It provides a consistent \proglang{C++} class hierarchy that
  maps various types of \proglang{R} objects (vectors, functions,
  environments, ...) to dedicated \proglang{C++} classes. Object interchange
  between \proglang{R} and \proglang{C++} is managed by simple, flexible and
  extensible concepts which include broad support for popular \proglang{C++}
  idioms from the Standard Template Library (STL). Using the \pkg{inline}
  package, \proglang{C++} code can be compiled, linked and loaded on the fly.
  Flexible error and exception code handling is provided.  \pkg{Rcpp}
  substantially lowers the barrier for programmers wanting to combine
  \proglang{C++} code with \proglang{R}.
  
  We discuss and motivate the two APIs provided by \pkg{Rcpp}: the older
  `classic' API introduced with the early releases of the package, and the
  `new' API that results from a recent redesign. We provided simple examples
  that show how \pkg{Rcpp} improves upon the standard \proglang{R} API,
  demonstrate performance implications of different program designs and show
  how \proglang{R} can take advantage of modern \proglang{C++} programming
  techniques, including template metaprogramming (TMP).  
  %
  The combination of modern \proglang{C++} together with the interactive
  environment provided by \proglang{R} creates a very compelling combination
  for statistical programming.
  
  \noindent \textbf{Keywords:} Foreign function interface, 
  \proglang{R}, \proglang{C++}, Standard Template Library (STL)
\end{abstract}
\end{document}

%%% Local Variables: 
%%% mode: latex
%%% TeX-master: t
%%% End: 
