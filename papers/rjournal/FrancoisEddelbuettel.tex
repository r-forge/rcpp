\title{Mesh R and C++ with Rcpp}
\author{by Romain Franc\c{c}ois and Dirk Eddelbuettel}

\maketitle

\abstract{
The \pkg{Rcpp} package provides a consistent and comprehensive set 
of C++ classes designed to ease coupling of C++ highly efficient code
with R. The \code{RObject} class assumes the responsability of 
protecting and releasing its encapsulated R object (\code{SEXP})
from garbage collection. The \code{wrap} set of functions allows
wrapping many C++ built-in types and data structures from the standard
template library into R objects. Similarly, the \code{as} set of 
templated functions allows conversion of R objects back into C++
types, such as \code{std::string}. With recent additions to the 
\pkg{inline} package \citep{cran:inline}, 
C++ code using the classes of the 
\pkg{Rcpp} package can be inlined, compiled, loaded and wrapped 
into an R function without leaving the R console. 
This article reviews some of the design choices of the
\pkg{Rcpp} package, in particular with respect to existing solutions
that deal with coupling R and C++ and shows several use cases.
}

Writing R Extensions \citep{R:exts} provides extensive documentation about the 
various ways to couple R with code written in C. 
Writing such code requires both expertise and discipline from the 
programmer. Discipline, with a large amount of bookkeeping 
duties around the \code{PROTECT}/\code{UNPROTECT} dance one 
has to master the steps. Expertise, to learn and use efficiently 
the set of macros offered by R headers. 

The \pkg{Rcpp} package makes extensive use of C++ features (encapsulation, 
constructors, destructors, operator overloading, templates) in order
to hide the complexity of the R API --- without losing its 
efficiency --- under the carpet of object orientation. 

\section{\pkg{Rcpp} C++ classes}

\subsection{The RObject class}

The \code{RObject} class is the base class of all objects in the 
API of the \pkg{Rcpp} package. An \code{RObject} has only one 
data member, the protected \code{SEXP} it encapsulates. 
The \code{RObject} treats the \code{SEXP} as a resource, following the
RAII (resource acquisition is initialization) pattern. As long as the 
\code{RObject} instance is alive, its underlying \code{SEXP} remains 
protected from garbage collection. When the \code{RObject} goes out 
of scope (function return, exceptions), it removes the protection so that 
if the \code{SEXP} is not otherwise protected it becomes subject to 
garbage collection. 

Garbage collection is only mentionned here to illustrate the basic design
of the \code{RObject} class, the user of \pkg{Rcpp} need not to concern 
himself/herself with such matters and can instead focus on the problem
that he/she is solving.

The \code{RObject} class also defines a set of member functions that
can be used on any R object, regardless of its type.

\begin{center}
\begin{small}
\begin{tabular}{cc}
method & action \\
\hline
\code{isNULL} & is the object \code{NULL}\\
\hline
\code{attributeNames} & the names of its attributes\\
\code{hasAttribute} & does it have a given attribute\\
\code{attr} & retrieve or set an attribute \\
\hline
\code{isS4} & is it an S4 object \\
\code{hasSlot} & if S4, does it have the given slot\\
\code{slot} & retrieve a given slot \\
\hline
\end{tabular}
\end{small} 
\end{center}

\subsection{Derived classes}

Internally, an R object must have one type amongst the set of 
predefined types, commonly referred to as SEXP types. R internals
\citep{R:ints} documents the various types. \pkg{Rcpp} associates
a C++ class for most SEXP types.
        
\begin{center}
\begin{small}
\begin{tabular}{ccc}
SEXP type &  \pkg{Rcpp} class \\
\hline 
\code{NILSXP} &  	\\
\code{SYMSXP} &	 \code{Symbol} \\
\code{LISTSXP} & \code{Pairlist} \\
\code{CLOSXP} &	 \code{Function} \\
\code{ENVSXP} &	 \code{Environment} \\
\code{PROMSXP} & \code{Promise} \\
\code{LANGSXP} & \code{Language} \\
\code{SPECIALSXP} & \code{Function} \\
\code{BUILTINSXP} & \code{Function} \\
\code{CHARSXP} & \\
\code{LGLSXP} &	 \code{LogicalVector} \\
\code{INTSXP} &	 \code{IntegerVector} \\
\code{REALSXP} & \code{NumericVector} \\
\code{CPLXSXP} & \code{ComplexVector}\\
\code{STRSXP} &	 \code{CharacterVector} \\
\code{DOTSXP} &	 \code{Pairlist} \\
\code{ANYSXP} &	 \\
\code{VECSXP} &	 \code{List} \\
\code{EXPRSXP} & \code{ExpressionVector}\\
\code{BCODESXP} & \\
\code{EXTPTRSXP} & \code{XPtr<T>}\\
\code{WEAKREFSXP} & \code{WeakReference}\\
\code{RAWSXP} &	 \code{RawVector}\\
\code{S4SXP} & \\
\hline
\end{tabular}
\end{small}
\end{center}

Some types do not have their own C++ class. \code{NILSXP} and 
\code{S4SXP} have their functionality covered by the \code{RObject}
class, \code{ANYSXP} is just a placeholder to facilitate S4 dispatch 
and no object in R has this type and \code{BCODESXP} is not currently 
used.

Each class contains functionality that is relevant to the R object
that it encapsulates. For example \code{Environment} contains 
member methods to query the list of objects in the associated environment, 
classes with the \code{Vector} overload the \code{operator[]} in order
to extract/modify values at the given position in the vector, ...

The rest of this section presents example uses of \pkg{Rcpp} classes. 

\subsection{numeric vector}

The following code snippet is extracted from Writing R extensions
\citep{R:exts}. It creates a \code{numeric} vector of two elements 
and assigns some values to it. 

\begin{example}
#include <R.h>
#include <Rinternals.h>

SEXP ab;
  ....
PROTECT(ab = allocVector(REALSXP, 2));
REAL(ab)[0] = 123.45;
REAL(ab)[1] = 67.89;
UNPROTECT(1);
\end{example}

Although this is one of the simplest examples in Writing R extensions, 
it seems verbose and it is not trivial at first sight what is happening.
\begin{itemize}
\item \code{allocVector} is used to allocate memory. We must supply to it 
the type of data (\code{REALSXP}) and the number of elements.
\item once allocated, the \code{ab} object must be protected from
garbage collection. Since the garbage collector can happen at any time, 
not protecting an object means its memory might be reclaimed before we are
finished with it.
\item The \code{REAL} macro returns a pointer to the beginning of the 
actual array. 
\end{itemize}

Using the \code{Rcpp::NumericVector}, the code can be rewritten: 

\begin{example}
#include <Rcpp.h>
using namespace Rcpp;
NumericVector ab(2) ;
ab[0] = 123.45;
ab[1] = 67.89;
\end{example}

The code contains much less idiomatic decorations. Here are the steps involved: 
\begin{itemize}
\item The \code{NumericVector} constructor is given the number
of elements the vector contains (2), this hides a call to the 
\code{allocVector} we saw previously. 
\item Also hidden is protection of the 
object from garbage collection, which is a behavior that \code{NumericVector}
inherits from \code{RObject}
\item values are assigned to the first and second elements of the vector. 
This is achieved \code{NumericVector} overloads the \code{operator[]}.
\end{itemize}

With recent compilers (e.g. GCC >= 4.4) implementing the forthcoming 
C++ standard (C++0x), the previous code may even be reduced 
to the following :

\begin{example}
#include <Rcpp.h>
using namespace Rcpp;
NumericVector ab = {123.45, 67.89};
\end{example}

\subsection{character vectors}

A second example deals with character vectors and emulates this R code

\begin{example}
> x <- c("foo", "bar")
\end{example}

Using the traditional R API, the vector can be allocated and filled as such:

\begin{example}
SEXP ab;
PROTECT(ab = allocVector(STRSXP, 2));
SET_STRING_ELT( ab, 0, mkChar("foo") );
SET_STRING_ELT( ab, 1, mkChar("bar") );
UNPROTECT(1);
\end{example}

Using the \pkg{Rcpp::CharacterVector} class, we can express this code as : 

\begin{example}
CharacterVector ab(2) ;
ab[0] = "foo" ;
ab[1] = "bar" ;
\end{example}

Additionally, if C++0x initializer list is implemented by the compiler, the 
code can be trimmed to the essential :

\begin{example}
CharacterVector ab = {"foo","bar"};
\end{example}


\section{wrap and as}

Besides classes, the \pkg{Rcpp} package also contains utilities allowing
conversion from R objects to C++ types and vice-versa. Through 
polymorphism, the \code{wrap} set of functions can be used to wrap 
some data structure into an \code{RObject} instance. In total, the 
\pkg{Rcpp} defines 23 different \code{wrap} functions, listed below :

\begin{small}
\begin{center}
\begin{tabular}{cc}
C++ type & \pkg{Rcpp} type \\
\hline
\code{SEXP} & $\star$ \\
\hline
\code{bool} & \code{LogicalVector} \\
\code{double} & \code{NumericVector}  \\
\code{int} & \code{IntegerVector}  \\
\code{size\_t} & \code{IntegerVector}  \\
\code{unsigned char} & \code{RawVector}  \\
\code{string} & \code{CharacterVector}  \\
\code{char*} & \code{CharacterVector}  \\
\hline
\code{vector<int>} & \code{IntegerVector}  \\
\code{vector<double>} & \code{NumericVector}  \\
\code{vector<unsigned char>} & \code{RawVector}  \\
\code{vector<bool>} & \code{LogicalVector}  \\
\code{vector<string>} & \code{CharacterVector}  \\
\hline
\code{set<int>} & \code{IntegerVector}  \\
\code{set<double>} & \code{NumericVector}  \\
\code{set<unsigned char>} & \code{RawVector}  \\
\code{set<string>} & \code{CharacterVector}  \\
\hline
\code{initializer\_list<int>} & \code{IntegerVector}  \\
\code{initializer\_list<double>} & \code{NumericVector}  \\
\code{initializer\_list<unsigned char>} & \code{RawVector}  \\
\code{initializer\_list<bool>} & \code{LogicalVector}  \\
\code{initializer\_list<string>} & \code{CharacterVector}  \\
\code{initializer\_list<RObject>} & \code{List} \\
\hline
\end{tabular}
\begin{small}$\star$ : depends on the type of the \code{SEXP}\end{small}
\end{center}
\end{small}

Here are a few examples of \code{wrap} calls: 

\begin{example}
LogicalVector x1 = wrap( false ); 
IntegerVector x2 = wrap( 1 ) ;    

vector<double> v ; 
v.push_back(0.0); v.push_back( 1.0 ); 
NumericVector x3 = wrap( v ) ;  

// initializer list (only on GCC >= 4.4)
LogicalVector x4 = wrap( \{ false, true\} );
CharacterVector x5 = wrap( \{"foo", "bar"\} );
\end{example}

Similarly, converting an R object to a C++ standard type is implemented
by variations on the \code{as} template function. In this case, we must 
use the angle brackets to specify which version of as we want to use. 

\begin{example}
bool x = as<bool>(x) ;
double x = as<double>(x) ;
vector<int> x = as< vector<int> >(x) ;
\end{example}

\section{external pointers}

factor this out from romain's blog

\section{inline code}

dirk ?

\section{others}

CXXR
Rserve C++ client
RcppTemplate
rcppbind
...

\section{Rcpp vintage api}


\section{Summary}





\bibliography{FrancoisEddelbuettel}

\address{Romain Fran\c{c}ois\\
  Professionnal R Enthusiast\\
  3 rue Emile Bonnet, 34 090 Montpellier\\
  FRANCE}\\
\email{francoisromain@free.fr}

\address{Dirk Eddelbuettel\\
  Affiliation\\
  Address\\
  Country}\\
\email{edd@debian.org}

