\title{Capitalized Title Here}
\author{by Romain Franc\c{c}ois and Dirk Eddelbuettel}

\maketitle

\abstract{
An abstract of less than 150 words.
}

Introductory section which may include references in parentheses, say
\citep{R:Ihaka+Gentleman:1996} or cite a reference such as
\citet{R:Ihaka+Gentleman:1996} in the text.

\section{Section title in sentence case}

This section may contain a figure such as Figure \ref{figure:onecolfig}.

\begin{figure}
\vspace*{.1in}
\framebox[\textwidth]{\hfill \raisebox{-.45in}{\rule{0in}{1in}}
                      A picture goes here \hfill}
\caption{\label{figure:onecolfig}
A normal figure only occupies one column.}
\end{figure}

\section{Another section}

There will likely be several sections, perhaps including code snippets, such
as
\begin{example}
  x <- 1:10
  result <- myFunction(x)
\end{example}

\section{Summary}

This file is only a basic article template. For full details of \emph{The R Journal}
style and information on how to prepare your article for submission, see the
\href{http://journal.r-project.org/latex/RJauthorguide.pdf}{Instructions for Authors}.

%\bibliography{example}

\begin{thebibliography}{1}
\expandafter\ifx\csname natexlab\endcsname\relax\def\natexlab#1{#1}\fi
\expandafter\ifx\csname url\endcsname\relax
  \def\url#1{{\tt #1}}\fi

\bibitem[Ihaka and Gentleman(1996)]{R:Ihaka+Gentleman:1996}
R.~Ihaka and R.~Gentleman.
\newblock R: A language for data analysis and graphics.
\newblock {\em Journal of Computational and Graphical Statistics}, 5\penalty0
  (3):\penalty0 299--314, 1996.
\newblock URL \url{http://www.amstat.org/publications/jcgs/}.

\end{thebibliography}

\address{Romain Fran\c{c}ois\\
  Professionnal R Enthusiast\\
  3 rue Emile Bonnet, 34 090 Montpellier\\
  FRANCE}\\
\email{francoisromain@free.fr}

\address{Dirk Eddelbuettel\\
  Affiliation\\
  Address\\
  Country}\\
\email{edd@debian.org}

