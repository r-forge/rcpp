%% Emacs please consider this:  -*- mode: latex; TeX-master: "RJwrapper.tex"; -*-

\title{Rcpp: Seamless R and C++ integration}
\author{by Dirk Eddelbuettel and Romain Franc\c{c}ois}

\maketitle

\abstract{TBD}

\section{Introduction}

TBD, probably close to last

One idea: call 'classic Rcpp' a \textsl{vertical} approach as it is chiefly
concerned with getting data from R to C++ and back from C++ to R. On the
other hand, 'new Rcpp' is more \textsl{horizontal as well as vertical} as it
also significantly eases access from C++ itself (and to the C++
representation of R objects).  Does that make sense?

\section{Overview}

% [Romain:] the overview is really messy and probably 
% needs a complete rewrite when all other sections are finished
% [Dirk:] Agreed, see Gelman piece. We need more meat on the bones first.

The \pkg{Rcpp} package provides a consistent and comprehensive set 
of C++ classes designed to ease coupling of C++ code
with R. The \code{RObject} class is responsible for 
protecting and releasing its encapsulated R object (\code{SEXP})
from garbage collection. The \code{wrap} set of functions allows
wrapping many C++ built-in types and data structures from the standard
template library into R objects. Similarly, the \code{as} set of 
templated functions allows conversion of R objects back into C++
types, such as \code{std::string}. With recent additions to the 
\pkg{inline} package \citep{cran:inline}, 
C++ code using the classes of the 
\pkg{Rcpp} package can be inlined, compiled, loaded and wrapped 
into an R function without leaving the R console. 
This article reviews some of the design choices of the
\pkg{Rcpp} package, in particular with respect to existing solutions
that deal with coupling R and C++ and shows several use cases.

Writing R Extensions \citep{R:exts} provides extensive 
documentation about the ways to couple R with code written in C. 
Writing such code requires both expertise and discipline from the 
programmer.

The \pkg{Rcpp} package makes extensive use of C++ features (encapsulation, 
constructors, destructors, operator overloading, templates) in order
to hide the complexity of the R API --- without losing its 
efficiency --- under the carpet of object orientation. In addition, 
\pkg{Rcpp} takes advantage of some features of the forthcoming \code{C++0x} 
standard, already supported by recent versions of the GCC.

\subsection{Historical Context}

\pkg{Rcpp} first appeared in 2005 as a contribution to the \pkg{RQuantLib}
package \citep{eddelbuettelkhan09:rquantlib} before being released as a CRAN
package in early 2006. Several releases followed in quick succession; all of
these were under the name \pkg{Rcpp}. The package was then renamed to
\pkg{RcppTemplate} and several more releases followed during 2006 under the
new name.  However, no new releases or updates were made during 2007, 2008
and most of 2009.

Given the continued use of the package, it was revived using the former name
\pkg{Rcpp}. New releases started in November 2008 which include an improved
build and distribution process, additional documentation, and new
functionality---while retaining the existing interface.  This constitutes the
`classic \pkg{Rcpp}' interface (described in the next section)
which will be provided for the forseeable future.

Yet C++ coding standards continued to evolved. So, starting in late 2009 the
codebase was significantly extended and numerous new features were added.
Several of these are described below in the section on the the `New
\pkg{Rcpp}' interface which we also intend to support going forward.

\subsection{Comparison}

Integration of C++ and R has been addressed by several authors; the earliest
published reference is probably \cite{batesdebroy01:cppclasses}.
An unpublished paper by \cite{javagailemanly07:r_cpp} expresses several ideas
that are close to some of our approaches, though not yet fully fleshed out.
%
The \pkg{Rserve} package \citep{cran:Rserve} was another early approach,
going back to 2002. On the server side, \pkg{Rserve} translates R data
structures into a binary serialization format and uses TCP/IP for
transfer. On the client side, objects are reconstructed as instances of Java
or C++ classes that emulate the structure of R objects. 

The packages \pkg{rcppbind} \citep{liang08:rcppbind}, \pkg{RAbstraction}
\citep{armstrong09:RAbstraction} and \pkg{RObjects}
\citep{armstrong09:RObjects} are all implemented using C++ templates.
However, neither one has matured to the point of a CRAN release and it
unclear how much usage these packages are seeing beyond their own authors.
%
CXXR \citep{runnalls09:cxxr} comes to this topic from the other side: 
its aim is to completely refactor R on a stronger C++ foundation. 
CXXR is therefore concerned with all aspects of the R interpreter,
REPL loop, threading --- and object interchange between R and C++ is but one part.
%
Another slightly different angle is offered by
\cite{templelang09:rgcctranslationunit} who uses compiler output for
references on the code in order to add bindings and wrappers.
%
Lastly, the \pkg{RcppTemplate} package \citep{samperi09:rcpptemplate}
recently introduced a few new ideas yet decided to break with the
`classic \pkg{Rcpp}' API.

A critical comparison of these packages that addresses relevant aspects such
API features, performance, useability and documentation would be a welcome
addition to the literature, but is beyond the scope of this article.

\section{Classic Rcpp}
\label{sec:classic_rcpp}

% [Romain:] Why 'at least initial'
% [Dirk:] For 'Classic Rcpp'
% [Romain:] I'd argue it is still the case with the new api
The core focus of \pkg{Rcpp}---particularly for the earlier API described in
this section---has always been on allowing the programmer to add C++-based
functions. We use this term in the standard mathematical sense of providing
results (output) given a set of parameters or data (input). This was
facilitated from the earliest releases using C++ classes for receiving
various types of R objects, converting them to C++ objects and allowing the
programmer to return the results to R with relative use. 

This API therefore supports two typical use cases. First, one can think of
replacing existing R code with equivalent C++ code in order to reap
performance gains.  This case can be conceptually easy as there may not be
(built- or run-time) dependencies on other C or C++ libraries.  It typically
involves setting up data and parameters---the right-hand side components of a
function call---before making the call in order to provide the result that is
to be assigned to the left-hand side. Second, \pkg{Rcpp} facilitates calling
functions provided by other libraries. The use resembles the first case: data
and parameters are passed via \pkg{Rcpp} to a function set-up to call code
from an external library.  

An illustration can be provided using the time-tested example of a
convolution of two vectors. This example is shown in sections 5.2 (for the
\code{.C()} interface) and 5.9 (for the \code{.Call()} interface) of 'Writing
R Extensions' \citep{R:exts}. We have rewritten it here using classes of the
classic \pkg{Rcpp} API:

\begin{example}
#include <Rcpp.h>

RcppExport SEXP convolve2cpp(SEXP a, SEXP b) \{
  RcppVector<double> xa(a);
  RcppVector<double> xb(b);
  int nab = xa.size() + xb.size() - 1;

  RcppVector<double> xab(nab);
  for (int i = 0; i < nab; i++) xab(i) = 0.0;

  for (int i = 0; i < xa.size(); i++)
    for (int j = 0; j < xb.size(); j++) 
       xab(i + j) += xa(i) * xb(j);

  RcppResultSet rs;
  rs.add("ab", xab);
  return rs.getReturnList();
\}
\end{example}

We can highlight several aspects. First, only a single header file
\code{Rcpp.h} is needed to use the \pkg{Rcpp} API.  Second, given two
\code{SEXP} types---the bread-and-butter of all internal R programming---a
third is returned.  Third, both inputs are converted to C++ vector types that
are \textsl{templated} (meaning that a type-indepedent framework can be
applied to create actual vectors of the specified type). Here a standard \code{double}
type is used to create a vector of doubles from the template type.
% [Romain:] I think the previous sentence is confusing, one might think
% that the same vector can hold int and double
% [Dirk:] Better?
% [Romain:] I think so, maybe the (...) should be a footnote
% [Dirk:] Sorry, which '(...)' ?
% [Romain:] (which means ... base types)
% [Dirk:] Ah. Better now? 
Fourth, the usefulness off these classes can be seen when we query the
vectors directly for their size---using the \code{size} member function---in
order to reserved a new result type of appropriate length whereas use based
on C arrays would have required additional parameters for the length of
vectors $a$ and $b$, leaving open the possibility of mismatches between the
actual length and the length reported by the programmer.  Fifth, the
computation itself is straightforward embedded looping just as in the
original examples in the 'Writing R Extensions' manual \citep{R:exts}.
Sixth, a return type (\code{RcppResultSet}) is then prepared as a named
object (something that should be familiar to R programmers) which is then
converted to a list object that is returned.  We should note that the
\code{RcppResultSet} permits the return of numerous (named) objects which can
also be of different types.

We argue that this usage is already easier to read, write and debug than the
C macro-based approach supported by R itself. Possible performance issues and
other potentual limitations will be discussed throughout the article and
reviewed at the end.

\section{Using code `inline'}

Extending R with compiled code also needs to address how to reliably compile,
link and load the code.  While using a package is preferable in the long run,
it may be to heavy a framework for quick explorations.  An alternative is
provided by the \pkg{inline} package \citep{cran:inline} which compiles,
links and loads a C, C++ or Fortran function---directly from the R prompt
using a simple function \code{cfunction}.  It was recently extended to work
with \pkg{Rcpp} by allowing for the use of additional header files and
libraries. This works particularly well with the \pkg{Rcpp} package where
headers and the library are automatically found if the appropriate option
\code{Rcpp} to \texttt{cfunction} is set to true.

% [Romain] : the next paragraph is very confusing
% [Dirk] Is this better?
% [Romain] Not sure. It seems to be only readable backwards. what about a 
%          separate section before 'inline code' just about this
% 
%          it might also be useful to show a quick example of inlining
%          c++ code, for example say that we use it for our unit tests
%          and show an example unit test
% [Dirk] Done in last round
% [Romain] But this shows the old api !!! and the same code as above so that 
%          people get to see it twice. I'd prefer moving these bits after 
%          the new Rcpp api section and show new api code inlined
The use of \pkg{inline} is possible as \pkg{Rcpp} can be installed and
updated just like any other R package using \textsl{e.g.} the
\code{install.packages()} function for initial installation as well as
\code{update.packages()} for upgrades.  So even though R / C++ interfacing
would otherwise require source code, the \pkg{Rcpp} library is always provided
ready for use as a pre-built library through the CRAN package mechanism.

The library and header files provided by \pkg{Rcpp} for use by other packages
are installed along with the \pkg{Rcpp} package making it possible for
\pkg{Rcpp} to provide the appropriate \code{-I} and \code{-L} switches needed
for compilation and linking.  So internally, \pkg{inline} makes uses of the
two functions \code{Rcpp:::CxxFlags()} and \code{Rcpp:::LdFlags()} that
provide this information (and which are also used by \code{Makefiles} of
other packages).  Here, however, all this is done behind the scenes and the
user need not worry about compiler or linker options or settings.

The convolution example provided above now can be rewritten for use by
\pkg{inline} as shown here.  The function body is provided by character
variable \code{src}, the function header is defined by the argument
\code{signature}---and we only need to enable \code{Rcpp=TRUE} to obtain a
new function \code{fun} based on the C++ code in \code{src}:
\begin{example}
src <- '
  RcppVector<double> xa(a);
  RcppVector<double> xb(b);
  int nab = xa.size() + xb.size() - 1;

  RcppVector<double> xab(nab);
  for (int i = 0; i < nab; i++) xab(i) = 0.0;

  for (int i = 0; i < xa.size(); i++)
    for (int j = 0; j < xb.size(); j++)
       xab(i + j) += xa(i) * xb(j);

  RcppResultSet rs;
  rs.add("ab", xab);
  return rs.getReturnList();
';
fun <- cfunction(signature(a="numeric", 
                           b="numeric"),
                 src, Rcpp=TRUE)
\end{example}


\section{New \pkg{Rcpp} API}
\label{sec:new_rcpp}

Having discussed the `Classic Rcpp' API and its deployment in the previous
section, we now turn to the `New Rcpp'. The new API is a complete redesign
based on the usage experience of several years of Rcpp deployment, as well as
current C++ design approaches.

\subsection{Rcpp Class hierarchy}

The \code{Rcpp::RObject} class is the basic class of the new Rcpp api. 
An instance of the \code{RObject} class encapsulates an R object
(\code{SEXP}), exposes methods that are appropriate for all types 
of objects and transparently manage garbage collection.

The most important aspect of the \code{RObject} class is that it is 
a very thin wrapper around the \code{SEXP} it encapsulates, the 
\code{SEXP} is indeed the only data member of an \code{RObject}.

The \code{RObject} class takes advantage of the explicit life cyle of 
c++ objects to implement garbage collection of R objects. The 
\code{RObject} effectively treats its underlying \code{SEXP} as 
a resource. The constructor of the \code{RObject} class takes 
the necessary measures to guarantee that the underlying \code{SEXP}
is protected from the garbage collector, and the destructor
assumes the responsability to withdraw that protection. 

By assuming the entire responsability of garbage collection, \code{Rcpp}
relieves the programmer from writing boiler plate code to manage
the protection stack with \code{PROTECT} and \code{UNPROTECT} macros.

The \code{RObject} class defines a set of member functions that
can be used on any R object, regardless of its type. The member
functions \code{isNULL}, \code{isObject} and \code{isS4} can be 
used to query properties of the object. 

Regarding attributes, the member functions 
\code{attributeNames} can be used to retrieve the names of the attributes, 
the \code{hasAttribute} can be used to query the existence of an attribute and 
the \code{attr} can be used to either get the current value of an 
attribute, or set the value to some other object.

Similarly, the member functions \code{hasSlot} and \code{slot}
can be used to manage slots of an S4 object. These function throw 
c++ exceptions when used on objects that are not S4 objects. 

% example of using attr or slot ?
% mention proxy pattern ?

\subsection{Derived classes}

Internally, an R object must have one type amongst the set of 
predefined types, commonly referred to as SEXP types. R internals
\citep{R:ints} documents these various types. 
\pkg{Rcpp} associates a dedicated C++ class for most SEXP types.

Each class contains functionality that is relevant to the R object
that it encapsulates. For example \code{Rcpp::Environment} contains 
member functions to manage objects in the associated environment. 
Classes related to vectors (\code{IntegerVector}, \code{NumericVector}, 
\code{RawVector}, \code{LogicalVector}, \code{CharacterVector}, 
\code{GenericVector} and \code{ExpressionVector}) expose functionality
to extract and set values from the vectors, etc ...

The following sub sections present typical uses Rcpp classes in
comparison with the same code expressed using functions of the R api.

\subsection{numeric vector}

The following code snippet is extracted from Writing R extensions
\citep{R:exts}. It creates a \code{numeric} vector of two elements 
and assigns some values to it. 

\begin{example}
SEXP ab;
PROTECT(ab = allocVector(REALSXP, 2));
REAL(ab)[0] = 123.45;
REAL(ab)[1] = 67.89;
UNPROTECT(1);
\end{example}

Although this is one of the simplest examples in Writing R extensions, 
it seems verbose and it is not trivial at first sight what is happening.
\begin{itemize}
\item \code{allocVector} is used to allocate memory. We must supply to it 
the type of data (\code{REALSXP}) and the number of elements.
\item once allocated, the \code{ab} object must be protected from
garbage collection. Since the garbage collector can happen at any time, 
not protecting an object means its memory might be reclaimed before we are
finished with it.
\item The \code{REAL} macro returns a pointer to the beginning of the 
actual array; its indexing is does not resemble either R or C++.
\end{itemize}

Using the \code{Rcpp::NumericVector} class, the code can be rewritten: 

\begin{example}
Rcpp::NumericVector ab(2) ;
ab[0] = 123.45;
ab[1] = 67.89;
\end{example}

The code contains much less idiomatic decorations. Here are the steps involved: 
\begin{itemize}
\item The \code{NumericVector} constructor is given the number
of elements the vector contains (2), this hides a call to the 
\code{allocVector} we saw previously. 
\item Also hidden is protection of the 
object from garbage collection, which is a behavior that \code{NumericVector}
inherits from \code{RObject}
\item values are assigned to the first and second elements of the vector. 
This is achieved \code{NumericVector} overloads the \code{operator[]}.
\end{itemize}

With the most recent compilers (e.g. G++ >= 4.4) which already implement
parts of the forthcoming C++ standard (C++0x), the preceding code may even be
reduced to this:

\begin{example}
Rcpp::NumericVector ab = \{123.45, 67.89\};
\end{example}

\subsection{character vectors}

A second example deals with character vectors and emulates this R code

\begin{example}
> x <- c("foo", "bar")
\end{example}

Using the traditional R API, the vector can be allocated and filled as such:

\begin{example}
SEXP ab;
PROTECT(ab = allocVector(STRSXP, 2));
SET_STRING_ELT( ab, 0, mkChar("foo") );
SET_STRING_ELT( ab, 1, mkChar("bar") );
UNPROTECT(1);
\end{example}

Using the \pkg{Rcpp::CharacterVector} class, we can express this code as : 

\begin{example}
CharacterVector ab(2) ;
ab[0] = "foo" ;
ab[1] = "bar" ;
\end{example}

Additionally, if C++0x initializer list is implemented by the compiler, the 
code can be trimmed to the essential :

\begin{example}
CharacterVector ab = \{"foo","bar"\};
\end{example}

\section{Data interchange between R and C++}

In addition to classes, the \pkg{Rcpp} package contains two additional
functions to perform conversion of C++ objects to R objects and back. 

The C++ to R conversion is performed by the \code{Rcpp::wrap} templated 
function. It uses advanced template meta programming techniques
to convert a wide and extensible set of types and classes to the
most appropriate type of R object. \code{wrap} will 
currently handle these C++ types: 
\begin{itemize}
\item primitive types, \code{int}, \code{double}, ... are converted 
into R vectors of the appropriate type
\item \code{std::string} are converted to R character vectors
\item STL-like containers, e.g \code{std::vector<T>}, \code{std::list<T>}, 
are wrappable as long as the type they contain (T) is wrappable. 
\item STL-like maps, e.g. \code{std::map<std::string,T>}, 
which uses \code{std::string} for their keys, are wrappable as long as 
the type \code{T} is wrappable
\item any type that implements implicit conversion to \code{SEXP}, through the 
\code{operator SEXP()} are wrappable
\end{itemize}

In addition, the \code{wrap} template may be partially or fully specialized by
third party code to extend its capabilities. The design allow composition, 
so for example objects of the class
\code{std::vector< std::map<std::string,int> >} are wrappable. This is 
because \code{int} is wrappable (as a primitive type), consequently 
\code{std::map<std::string,int>} is wrappable (as an STL-like map of 
wrappable types keyed by strings, and therefore
\code{std::vector< std::map<std::string,int> >} is wrappable (as a 
STL-like container of wrappable objects). The example code below
illustrates this: 

\begin{example}
std::vector< std::map<std::string,int> > v ;

std::map< std::string, int > m1 ;
m1["foo"] = 1 ; m1["bar"] = 2 ;

std::map< std::string, int > m2 ;
m2["foo"] = 1 ; m2["bar"] = 2 ; m2["bling"] = 3 ;

v.push_back( m1) ;
v.push_back( m2) ;

wrap( v ) ;
\end{example}

The code creates a list of two named vectors, equal to the list that 
can be created by the following R code: 

\begin{example}
list( c( bar = 2L, foo = 1L) , c( bar = 2L, bling = 3L, foo = 1L) )
\end{example}

The reversed conversion is implemented by variations of the 
\code{Rcpp::as} template. \code{as} offers less flexibility and currently
handles conversion of R objects into primitive types (bool, int, std::string, ...), 
STL vectors of primitive types  (\code{std::vector<bool>}, 
\code{std::vector<double>}, etc ...) and arbitrary types that offer 
a constructor that takes a \code{SEXP}. In addition \code{as} can 
be fully or partially specialized to manage conversion of R data 
structures to third party types.

The converters offered by \code{wrap} and \code{as} provide a very 
useful framework to implement the logic of the code in terms of C++ 
data structures and then explicitely convert data back to R, ...

The converters are also used implicitely in various places in the 
\code{Rcpp} api. Consider the following code that uses the
\code{Rcpp::Environment} class to interchange data between C++ and R.

\begin{example}
# assuming the global environment contains 
# a variable 'x' that is a numeric vector
Rcpp::Environment global = Rcpp::Environment::global_env()

# extract a std::vector<double> from the global environment
std::vector<double> vx = global["x"] ;

# create a map<string,string>
std::map<std::string,std::string> map ;
map["foo"] = "oof" ;
map["bar"] = "rab" ;

# push the STL map to the global environment
global["y"] = map ;
\end{example}

In the first part of the example, \code{as} is used implicitely to convert
the object "x" from the global environment into an instance
of the \code{std::vector<double>} class. In the second part of the example, 
\code{wrap} is used implicitely to convert the object of class
\code{std::map<std::string,std::string>} into an R object, a named
character vector in this case.

\section{other examples}

The last example shows how to use \pkg{Rcpp} to emulate the R code below.

\begin{example}
> rnorm( 10L, sd = 100.0 )
\end{example}

The code can be expressed in several ways in \pkg{Rcpp}, the first version
shows the use of the \code{Environment} and \code{Function} classes. 

\begin{example}
Environment stats("package:stats") ;
Function rnorm = stats.get("rnorm") ;
return rnorm(10, Named("sd", 100.0) ) ;
\end{example}

We first pull out the \code{rnorm} function from the environment 
called \samp{package:stats} in the search path, then call the function
using syntax similar to calling the function in R. The \code{Rcpp::Named} 
class is an utility class that is used to emulate named arguments.

The second version shows the use of the \code{Language} class, which 
manage calls (LANGSXP). 

\begin{example}
Language call("rnorm", 10, Named("sd", 100 ) ) ;
call.eval() ;
\end{example}

In this version, we first create a call to the symbol "rnorm" and
evaluate the call in the global environment. In both cases, \code{wrap}
is used implicitely to convert \code{10} and \code{100} 
into R integer vectors. 

Using the R API, the first example, using the actual \code{rnorm} function,
translates to :

\begin{example}
SEXP stats = PROTECT( 
\ \ R_FindNamespace( mkString("stats") ) ) ;
SEXP rnorm = PROTECT( 
\ \ findVarInFrame( stats, install("rnorm") ) ) ;
SEXP call  = PROTECT( 
\ \ LCONS( rnorm, 
\ \ \ \ CONS(ScalarInteger(10), 
\ \ \ \ \ \ CONS(ScalarReal(100.0), R_NilValue)))) ;
SET_TAG( CDDR(call), install("sd") ) ;
SEXP res = PROTECT( eval( call, R_GlobalEnv ) );
UNPROTECT(4) ;
return res ;
\end{example}

and the second example, using the \samp{rnorm} symbol --- and therefore
involving potentially expensive implicit lookup in the search path ---
can be written as:

\begin{example}
SEXP call  = PROTECT( 
\ \ LCONS( install("rnorm"), 
\ \ \ \ CONS(ScalarInteger(10), 
\ \ \ \ \ \ CONS(ScalarReal(100.0), R_NilValue)))) ;
SET_TAG( CDDR(call), install("sd") ) ;
SEXP res = PROTECT( eval( call, R_GlobalEnv ) );
UNPROTECT(2) ;
return res ;
\end{example}

For more examples, the reader is invited to 
refer to the documentation included in \pkg{Rcpp}
as well as the many examples that the package contains as part of 
its unit tests. 

\section{Performance/Limitations}

In this section, we illustrate that C++ features come with a price
in terms of performance. As users of \pkg{Rcpp}, we do not want
to replace performance with comfort. 

As part of the redesign of \pkg{Rcpp}, data copy is kept to the
absolute minimum, the \code{RObject} class and all its derived
classes are just a container for a \code{SEXP}, we let R perform
all memory management and access data though the macros or functions
offered by the standard R API. In contrasts, some data structures
of the classic \pkg{Rcpp} interface such as the templated 
\code{RcppVector} used containers offered by the standard template
library to hold the data, requiring copy of the data 
from R to C++ and back.

In this section, we illustrate how to take advantage of \code{Rcpp} to get
the best of it. The classic Rcpp translation of the convolve example from
\cite{R:exts} appears in section~\ref{sec:classic_rcpp}.  With the new API,
the code can be written as shown below. The main difference is that the input
parameters are directly passed to types \code{Rcpp::NumericVector}, and that
the return vector is automatically converted to a \code{SEXP} type through 
implicit conversion.

\begin{example}
#include <Rcpp.h>

RcppExport SEXP convolve3cpp(SEXP a, SEXP b)\{
    Rcpp::NumericVector xa(a);
    Rcpp::NumericVector xb(b);
    int n_xa = xa.size() ;
    int n_xb = xb.size() ;
    int nab = n_xa + n_xb - 1;
    Rcpp::NumericVector xab(nab);

    for (int i = 0; i < nab; i++) xab[i] = 0.0;
    for (int i = 0; i < n_xa; i++)
        for (int j = 0; j < n_xb; j++) 
            xab[i + j] += xa[i] * xa[j];

    return xab ;
\}
\end{example}

The implementation of the \code{operator[]} is implemented as 
efficiently as possible, using inlining and caching, 
but the implementation above is however less efficient than the 
reference C imlementation described in \cite{R:exts}. 

In order to achieve maximulm effociency, the reference implementation
extracts the underlying array pointer : \code{double*} and works 
with pointer arithmetics, which is a built-in operation as opposed to 
calling the \code{operator[]} on a user-defined class which has to 
pay the price of object encapsulation.

Modelled after containers of the standard template library, 
the \code{NumericVector} class provides two member functions \code{begin}
and \code{end} that can use used to retrieve respectively 
the pointer to the first and past to end elements of the underlying array.
We can revisit the code to take advantage of this feature : 

\begin{example}
#include <Rcpp.h>

RcppExport SEXP convolve4cpp(SEXP a, SEXP b) \{
    Rcpp::NumericVector xa(a);
    Rcpp::NumericVector xb(b);
    int n_xa = xa.size() ;
    int n_xb = xb.size() ;
    int nab = n_xa + n_xb - 1;
    Rcpp::NumericVector xab(nab);
    
    double* pa = xa.begin() ;
    double* pb = xb.begin() ;
    double* pab = xab.begin() ;
    int i,j=0; 
    for (i = 0; i < nab; i++) pab[i] = 0.0;
    for (i = 0; i < n_xa; i++)
        for (j = 0; j < n_xb; j++) 
            pab[i + j] += pa[i] * pb[j];

    return xab ;
\}
\end{example}

The following timings show the time taken (in milliseconds) 
by 1000 replicates of each function with \code{a} and 
\code{b} containing 100 elements.

\begin{center}
\begin{tabular}{cc}
Method & elapsed time (ms) \\ 
\hline
R API & 34 \\
\code{RcppVector<double>} & 353 \\
\code{NumericVector::operator[]} & 55 \\
\code{NumericVector::begin} & 36 \\
\hline
\end{tabular}
\end{center}

% need to comment the results, give reasons why the RcppVector<double> is
% 10 times less efficient than the reference, show that 55-36 is the price for 
% encapsulation and say that the difference between 34 and 36 is not 
% significant

\section{Summary}

% The \code{Rcpp} package provides comprehensive set of C++
% classes aimed at significantly reducing the complexity and
% discipline involved in combining R with compiled code.
% 
% By assuming the responsibility of protection against garbage
% collection automatically and transparently and encapsulating R objects
% in C++ classes, \pkg{Rcpp} empowers the developper to concentrate on 
% the problem at hand instead of manually keeping track of 
% the \code{PROTECT}/\code{UNPROTECT} dance and without requiring 
% the expertise of knowing the details of the many macros and functions
% of the R internal API.
% 
% Evidently, C++ has a price and we have shown how to take advantage
% of \code{Rcpp} to reduce --- if not eliminate --- the overhead while
% significantly improving code clarity and maintainability. 


\bibliography{EddelbuettelFrancois}

\address{Dirk Eddelbuettel\\
  Debian Project\\
  Chicago, IL\\
  USA}\\
\email{edd@debian.org}

\address{Romain Fran\c{c}ois\\
  Professionnal R Enthusiast\\
  3 rue Emile Bonnet, 34 090 Montpellier\\
  FRANCE}\\
\email{francoisromain@free.fr}

