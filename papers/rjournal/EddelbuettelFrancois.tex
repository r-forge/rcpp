%% Emacs please consider this:  -*- mode: latex; TeX-master: "RJwrapper.tex"; -*-

\title{Rcpp: Seamless R and C++ integration}
\author{by Dirk Eddelbuettel and Romain Fran\c{c}ois}

\maketitle

\abstract{The \pkg{Rcpp} package simplifies integrating C++ code with R. It
  provides a consistent C++ class hierarchy that maps various types of R
  objects (vectors, functions, environments, ...) to dedicated C++
  classes. Object interchange between R and C++ is managed by
  simple, flexible and extensible concepts which include broad support for
  C++ STL idioms. C++ code can be compiled, linked and loaded on the fly.
  Flexible error and exception code handling is provided.  \pkg{Rcpp}
  substantially lowers the barrier for programmers wanting to combine
  C++ code with R.}
% [dirk] should we mention C as well?
% [romain]�no

\section{Introduction} 

R is an extensible system. The `Writing R Extensions' manual \citep{R:exts}
describes in detail how to augment R with compiled code,
focussing mostly on the C language. The R API described in `Writing R Extensions' is
based on a set of functions and macros operating on \code{SEXP}, the internal
representation of R objects.
%
In this article, we discuss the functionality of the \pkg{Rcpp}
package, which simplifies the usage of C++ code
in R. Combining R and C++ is not a new idea, so we start with
a short review of other approaches and give some historical
background on the development of \pkg{Rcpp}.

The \pkg{Rcpp} package combines two distinct
APIs. The first---which we call `classic \pkg{Rcpp} API'---exists since 
the first version of \pkg{Rcpp}. While still contained in the package to
ensure compatibility, its use is otherwise deprecated. All new development should
use the newer and richer second API. It is enclosed in the \code{Rcpp} C++ 
namespace, and corresponds to the newer redesigned codebase. 
This article highlights some of the key design and implementation choices of
the new API: lightweight encapsulation of R objects in C++ classes, automatic
garbage collection strategy, code inlining, data interchange between R and
C++ and error handling.

Several examples are included to illustrate the benefits of using \pkg{Rcpp}
as opposed to the traditional R API. Many more examples are available within
the package, both as explicit examples and as part of the numerous unit tests.

\subsection{Historical Context}

\pkg{Rcpp} first appeared in 2005 as a contribution (by Samperi) to the
\pkg{RQuantLib} package \citep{cran:rquantlib} before becoming a CRAN package
in early 2006. Several releases (all by Samperi) followed in quick succession
under the name \pkg{Rcpp}. The package was then renamed to
\pkg{RcppTemplate}; several more releases followed during 2006 under the new
name.  However, no new releases were made during 2007, 2008 or most of
2009. Following a few brief updates in late 2009, it was withdrawn from CRAN
for good.

Given the continued use of the package, Eddelbuettel decided to revitalize it. New
releases, using the initial name \pkg{Rcpp}, started in November 2008. These
already included an improved build and distribution process, additional
documentation, and new functionality---while retaining the existing
interface.  This constitutes the `classic \pkg{Rcpp}' interface 
(not described in this article) which will be maintained for the foreseeable future.

Yet C++ coding standards continued to evolve \citep{meyers:effectivecplusplus}.
In 2009, Eddelbuettel and Fran\c{c}ois started a significant redesign of the
codebase which added numerous new features.  Several of these are described
below in the section on the \pkg{Rcpp} API interface, as well as in the
eight vignettes included with the package. This new API is our current focus,
and we intend to both extend and support it in future development of the
package. 

\subsection{Comparison}

Integration of C++ and R has been addressed by several authors; the earliest
published reference is probably \cite{batesdebroy01:cppclasses}.
An unpublished paper by \cite{javagailemanly07:r_cpp} expresses several ideas
that are close to some of our approaches, though not yet fully fleshed out.
%
The \pkg{Rserve} package \citep{urbanek2003:rserve,cran:Rserve} was another early approach,
going back to 2002. On the server side, \pkg{Rserve} translates R data
structures into a binary serialization format and uses TCP/IP for
transfer. On the client side, objects are reconstructed as instances of Java
or C++ classes that emulate the structure of R objects. 

The packages \pkg{rcppbind} \citep{liang08:rcppbind}, \pkg{RAbstraction}
\citep{armstrong09:RAbstraction} and \pkg{RObjects}
\citep{armstrong09:RObjects} are all implemented using C++ templates.
However, neither has matured to the point of a CRAN release and it is
unclear how much usage these packages are seeing beyond their own authors.
%
CXXR \citep{runnalls09:cxxr} comes to this topic from the other side: 
its aim is to completely refactor R on a stronger C++ foundation. 
CXXR is therefore concerned with all aspects of the R interpreter,
REPL, threading---and object interchange between R and C++ is but one
part. A similar approach is discussed by \cite{templelang09:modestproposal}
who suggests making low-level internals extensible by package developers in
order to facilitate extending \R.
%
Another slightly different angle is offered by
\cite{templelang09:rgcctranslationunit} who uses compiler output for
references on the code in order to add bindings and wrappers.

\subsection{Rcpp Use Cases}  % or some such
\label{sec:classic_rcpp}

The core focus of \pkg{Rcpp} has always been on allowing the 
programmer to add C++-based functions. 
We use this term in the standard mathematical sense of providing
results (output) given a set of parameters or data (input). 
This was
facilitated from the earliest releases using C++ classes for receiving
various types of R objects, converting them to C++ objects and allowing the
programmer to return the results to R with relative ease. 

This API therefore supports two typical use cases. First, one can think of
replacing existing R code with equivalent C++ code in order to reap
performance gains.  This case is conceptually easy as there may not be
(built- or run-time) dependencies on other C or C++ libraries.  It typically
involves setting up data and parameters---the right-hand side components of a
function call---before making the call in order to provide the result that is
to be assigned to the left-hand side. Second, \pkg{Rcpp} facilitates calling
functions provided by other libraries. The use resembles the first case: data
and parameters are passed via \pkg{Rcpp} to a function set-up to call code
from an external library.  

\section{The \pkg{Rcpp} API}
\label{sec:new_rcpp}

More recently, the \pkg{Rcpp} API has been dramatically extended, leading to a 
complete redesign, based on the usage experience of several 
years of \pkg{Rcpp} deployment, needs from other projects, knowledge 
of the internal R API, as well as current C++ design approaches. 
This redesign of \pkg{Rcpp} was also motivated by the needs of other 
projects such as \pkg{RInside}  \citep{cran:rinside} for easy embedding 
of R in a C++ applications and \code{RProtoBuf} \citep{cran:rprotobuf} 
that interfaces with the Protocol Buffers library. 

\subsection{A First Example}

\begin{example}
#include <Rcpp.h>

RcppExport SEXP convolve3cpp(SEXP a, SEXP b) \{
    Rcpp::NumericVector xa(a);
    Rcpp::NumericVector xb(b);
    int n_xa = xa.size(), n_xb = xb.size();
    int nab = n_xa + n_xb - 1;
    Rcpp::NumericVector xab(nab);

    for (int i = 0; i < n_xa; i++)
        for (int j = 0; j < n_xb; j++) 
            xab[i + j] += xa[i] * xb[j];

    return xab;
\}
\end{example}

We can highlight several aspects. First, only a single header file
\code{Rcpp.h} is needed to use the \pkg{Rcpp} API.  Second, given two
\code{SEXP} types, a third is returned.  Third, both inputs are 
converted to C++ vector types provided by \pkg{Rcpp} (and we have more to say about these
conversions below).  Fourth, the
usefulness of these classes can be seen when we query the vectors directly
for their size---using the \code{size()} member function---in order to
reserve a new result type of appropriate length,  
and with the use of the 
\verb|operator[]| to extract and set individual elements of the vector. 
Fifth, the computation itself is
straightforward embedded looping just as in the original examples in the
'Writing R Extensions' manual \citep{R:exts}.  Sixth, the return conversion
is also automatic from the \code{NumericVector} to the \code{SEXP} type.

We argue that this usage is much easier to read, write and debug than the
C macro-based approach supported by R itself. 

% [dirk]  TODO maybe another sentence to tie into the next segment


\subsection{Rcpp Class hierarchy}

The \code{Rcpp::RObject} class is the basic class of the new \pkg{Rcpp} API. 
An instance of the \code{RObject} class encapsulates an R object
(\code{SEXP}), exposes methods that are appropriate for all types 
of objects and transparently manages garbage collection.

The most important aspect of the \code{RObject} class is that it is 
a very thin wrapper around the \code{SEXP} it encapsulates. The 
\code{SEXP} is indeed the only data member of an \code{RObject}. The 
\code{RObject} class does not interfere with the way R manages its 
memory, does not perform copies of the object into a suboptimal 
C++ representation, but instead merely acts as a proxy to the 
object it encapsulates so that methods applied to the \code{RObject}
instance are relayed back to the \code{SEXP} in terms of the standard
R API.

The \code{RObject} class takes advantage of the explicit life cycle of 
C++ objects to manage exposure of the underlying R object to the 
garbage collector. The \code{RObject} effectively treats 
its underlying \code{SEXP} as a resource.
The constructor of the \code{RObject} class takes 
the necessary measures to guarantee that the underlying \code{SEXP}
is protected from the garbage collector, and the destructor
assumes the responsibility to withdraw that protection. 

By assuming the entire responsibility of garbage collection, \code{Rcpp}
relieves the programmer from writing boiler plate code to manage
the protection stack with \code{PROTECT} and \code{UNPROTECT} macros.

The \code{RObject} class defines a set of member functions applicable
to any R object, regardless of its type. This ranges from 
querying properties of the object (\texttt{isNULL}, \texttt{isObject}, 
\texttt{isS4}), management of the attributes 
(\texttt{attributeNames}, \texttt{hasAttribute}, \texttt{attr}) and 
handling of slots\footnote{The member functions that deal with slots
are only applicable on S4 objects; otherwise an exception is thrown.} 
(\texttt{hasSlot}, \texttt{slot}).

\subsection{Derived classes}

Internally, an R object must have one type amongst the set of 
predefined types, commonly referred to as SEXP types. The `R Internals'
manual \citep{R:ints} documents these various types. 
\pkg{Rcpp} associates a dedicated C++ class for most SEXP types, 
therefore only exposes functionality that is relevant to the R object
that it encapsulates.

For example \code{Rcpp::Environment} contains 
member functions to manage objects in the associated environment. 
Similarly, classes related to vectors (\code{IntegerVector}, \code{NumericVector}, 
\code{RawVector}, \code{LogicalVector}, \code{CharacterVector}, 
\code{GenericVector} (also known as \code{List}) and \code{ExpressionVector}) 
expose functionality to extract and set values from the vectors.

The following sub-sections present typical uses of \pkg{Rcpp} classes in
comparison with the same code expressed using functions of the R API.

\subsection{Numeric vectors}

The following code snippet is taken from Writing R extensions
\citep{R:exts}. It creates a \code{numeric} vector of two elements 
and assigns some values to it. 

\begin{example}
SEXP ab;
PROTECT(ab = allocVector(REALSXP, 2));
REAL(ab)[0] = 123.45;
REAL(ab)[1] = 67.89;
UNPROTECT(1);
\end{example}

Although this is one of the simplest examples in Writing R extensions, 
it seems verbose and it is not obvious at first sight to understand what is happening.
Memory is allocated by \code{allocVector}; we must also supply it with
the type of data (\code{REALSXP}) and the number of elements.  Once
allocated, the \code{ab} object must be protected from garbage
collection.
Lastly, the \code{REAL} macro returns a pointer to the
beginning of the actual array; its indexing does not resemble either R or
C++.

Using the \code{Rcpp::NumericVector} class, the code can be rewritten: 

\begin{example}
Rcpp::NumericVector ab(2);
ab[0] = 123.45;
ab[1] = 67.89;
\end{example}

The code contains fewer idiomatic decorations. The \code{NumericVector}
constructor is given the number of elements the vector contains (2), this
hides a call to the \code{allocVector} we saw previously. Also hidden is
protection of the object from garbage collection, which is a behavior that
\code{NumericVector} inherits from \code{RObject}.  Values are assigned to
the first and second elements of the vector as \code{NumericVector} overloads
the \code{operator[]}.

The snippet can also be written more concisely using the \code{create}
static member function of the \code{NumericVector} class: 

\begin{example}
Rcpp::NumericVector ab = 
  Rcpp::NumericVector::create(123.45, 67.89);
\end{example}

It should be noted that although the copy constructor of the 
\code{NumericVector} class is used, it does not imply copies of the 
underlying array, only the \code{SEXP} is copied. 

\subsection{Character vectors}

A second example deals with character vectors and emulates this R code

\begin{example}
> c("foo", "bar")
\end{example}

Using the traditional R API, the vector can be allocated and filled as such:

\begin{example}
SEXP ab;
PROTECT(ab = allocVector(STRSXP, 2));
SET_STRING_ELT( ab, 0, mkChar("foo") );
SET_STRING_ELT( ab, 1, mkChar("bar") );
UNPROTECT(1);
\end{example}

This imposes on the programmer knowledge of \code{PROTECT}, \code{UNPROTECT}, 
\code{SEXP}, \code{allocVector}, \code{SET\_STRING\_ELT}, and \code{mkChar}. 

Using the \code{Rcpp::CharacterVector} class, we can express the same
code more concisely:

\begin{example}
CharacterVector ab(2);
ab[0] = "foo";
ab[1] = "bar";
\end{example}

\section{R and C++ data interchange}

In addition to classes, the \pkg{Rcpp} package contains two 
functions to perform conversion of C++ objects to R objects and back. 

\subsection{C++ to R : wrap}

The C++ to R conversion is performed by the \code{Rcpp::wrap} templated
function. It uses advanced template metaprogramming techniques\footnote{A
  discussion of template metaprogramming
  \citep{Vandevoorde+Josuttis:2003,Abrahams+Gurtovoy:2004} is beyond the
  scope of this article.}  to convert a wide and extensible set of types and
classes to the most appropriate type of R object. The signature of the
\code{wrap} template is:

\begin{example}
template <typename T> 
SEXP wrap(const T& object);
\end{example}

The templated function takes a reference to a `wrappable' 
object and converts this object into a \code{SEXP}, which is what R expects. 
Currently wrappable types are :
\begin{itemize}
\item primitive types: \code{int}, \code{double}, ... which are converted 
into the corresponding atomic R vectors;
\item \code{std::string} objects which are converted to R atomic character vectors;
\item STL containers such as \code{std::vector<T>} or \code{std::list<T>}, 
as long as the template parameter type \code{T} is itself wrappable;
\item STL maps which use \code{std::string} for keys 
(e.g. \code{std::map<std::string,T>}); as long as 
the type \code{T} is wrappable;
\item any type that implements implicit conversion to \code{SEXP} through the 
\code{operator SEXP()};
\item any type for which the \code{wrap} template is
fully specialized.
\end{itemize}

Wrappability of an object type is resolved at compile time using 
modern techniques of template meta programming and class traits. The 
\code{Rcpp-extending} vignette discusses in depth how to extend \code{wrap}
to third-party types. The \pkg{RcppArmadillo} package
\citep*{cran:rcpparmadillo} features several examples.

The following code snippet illustrates that the design allows
composition:

\begin{example}
std::vector< std::map<std::string,int> > v;
std::map< std::string, int > m1;
std::map< std::string, int > m2;

m1["foo"] = 1; m1["bar"] = 2;
m2["foo"] = 1; m2["bar"] = 2; m2["baz"] = 3;

v.push_back( m1);
v.push_back( m2);
Rcpp::wrap( v );
\end{example}

The code creates a list of two named vectors, equal to the 
result of this R statement:

\begin{example}
list( c( bar = 2L, foo = 1L) , 
      c( bar = 2L, baz = 3L, foo = 1L) )
\end{example}

\subsection{R to C++ : as}

The reversed conversion is implemented by variations of the 
\code{Rcpp::as} template whose signature is:
\begin{example}
template <typename T> 
T as(SEXP x) throw(not_compatible);
\end{example}

It offers less flexibility and currently
handles conversion of R objects into primitive types (bool, int, std::string, ...), 
STL vectors of primitive types  (\code{std::vector<bool>}, 
\code{std::vector<double>}, etc ...) and arbitrary types that offer 
a constructor that takes a \code{SEXP}. In addition \code{as} can 
be fully or partially specialized to manage conversion of R data 
structures to third-party types as can be seen for example in the
\pkg{RcppArmadillo} package which eases transfer to R matrices and vectors to
the optimised data structures in the Armadillo linear algebra library \citep{Armadillo}.


\subsection{Implicit use of converters}

The converters offered by \code{wrap} and \code{as} provide a very 
useful framework to implement the logic of the code in terms of C++ 
data structures and then explicitly convert data back to R. 

In addition, the converters are also used implicitly
in various places in the \code{Rcpp} API. 
Consider the following code that uses the \code{Rcpp::Environment} class to 
interchange data between C++ and R.

\begin{example}
// assuming the global environment contains 
// a variable 'x' that is a numeric vector
Rcpp::Environment global = 
\ \ \ \ Rcpp::Environment::global_env();

// extract a std::vector<double> from 
// the global environment
std::vector<double> vx = global["x"];

// create a map<string,string>
std::map<std::string,std::string> map;
map["foo"] = "oof";
map["bar"] = "rab";

// push the STL map to R
global["y"] = map;
\end{example}

In the first part of the example, the code extracts a 
\code{std::vector<double>} from the global environment. In order to achieve this, 
the \code{operator[]}  of \code{Environment} uses the proxy pattern 
\citep{meyers:moreeffectivecplusplus}
to distinguish between left hand side (LHS) and right hand side (RHS) use. 
%
The output of the operator is an instance of the nested class
\code{Environment::Binding}, which defines a templated implicit conversion 
operator that allows a \code{Binding} to be assigned to any type that 
\code{Rcpp::as} is able to handle. 

In the second part of the example, LHS use of the \code{Binding} instance is 
implemented through its assignment operator, which is also templated and uses
\code{Rcpp::wrap} to perform the conversion to a \code{SEXP} that can be 
assigned to the requested symbol in the global environment. 

The same mechanism is used throughout the API. Examples include access/modification
of object attributes, slots, elements of generic vectors (lists), 
function arguments, nodes of dotted pair lists and language calls and more. 

\section{Function calls}

\begin{table*}[t]
  \begin{minipage}[t]{0.465\linewidth}
    \centering{\underline{Environment: Using the \pkg{Rcpp} API}}
    \begin{example}
Environment stats("package:stats");
Function rnorm = stats["rnorm"];
return rnorm(10, Named("sd", 100.0));
    \end{example}
  \end{minipage}
  \begin{minipage}{0.06\linewidth}
    \phantom{XXX}
  \end{minipage}
  \begin{minipage}[t]{0.465\linewidth}
    \centering{\underline{Environment: Using the R API}}
    \begin{example}
SEXP stats = PROTECT( 
\ \ R_FindNamespace( mkString("stats")));
SEXP rnorm = PROTECT( 
\ \ findVarInFrame( stats, install("rnorm")));
SEXP call  = PROTECT( 
\ \ LCONS( rnorm, 
\ \ \ \ CONS(ScalarInteger(10), 
\ \ \ \ \ \ CONS(ScalarReal(100.0),R_NilValue))));
SET_TAG( CDDR(call), install("sd") );
SEXP res = PROTECT(eval(call, R_GlobalEnv));
UNPROTECT(4);
return res;
    \end{example}
  \end{minipage}


  \begin{minipage}[t]{0.465\linewidth}
    \centering{\underline{Language: Using the \pkg{Rcpp} API}}
    \begin{example}
Language call("rnorm",10,Named("sd",100.0));
return call.eval();
    \end{example}
  \end{minipage}
  \begin{minipage}{0.06\linewidth}
    \phantom{XXX}
  \end{minipage}
  \begin{minipage}[t]{0.465\linewidth}
    \centering{\underline{Language: Using the R API}}
    \begin{example}
SEXP call  = PROTECT( 
\ \ LCONS( install("rnorm"), 
\ \ \ \ CONS(ScalarInteger(10), 
\ \ \ \ \ \ CONS(ScalarReal(100.0),R_NilValue))));
SET_TAG( CDDR(call), install("sd") );
SEXP res = PROTECT(eval(call, R_GlobalEnv));
UNPROTECT(2);
return res;
    \end{example}
  \end{minipage}
  \caption{\pkg{Rcpp} versus the R API: Four ways of calling \code{rnorm(10L, sd=100)} in C / C++. 
  We have removed the \code{Rcpp::} prefix from the
  examples for readability; this corresponds to adding a directive
  \texttt{using namespace Rcpp;} in the code}
  \label{fig:rnormCode}
\end{table*}

The next example shows how to use \pkg{Rcpp} to emulate the R code
\code{rnorm(10L, sd=100.0)}.
%
As shown in Table~\ref{fig:rnormCode}, the code can be expressed in several
ways in either \pkg{Rcpp} or the standard R API. The first version shows the
use of the \code{Environment} and \code{Function} classes by
\pkg{Rcpp}.
The second version shows the use of the \code{Language} class, which 
manage calls (LANGSXP). 
For comparison, we also show both versions using the standard R API.

This example illustrates that the \pkg{Rcpp} API permits us to work with code
that is easier to read, write and maintain. More examples are available as
part of the documentation included in the \pkg{Rcpp} package, as well as
among its over seven hundred and seventy unit tests.

\section{Using code `inline'}
\label{sec:inline}

Extending R with compiled code also needs to address how to reliably compile,
link and load the code.  While using a package is preferable in the long run,
it may be too involved for quick explorations. An alternative is
provided by the \pkg{inline} package \citep{cran:inline} which compiles,
links and loads a C, C++ or Fortran function---directly from the R prompt
using a simple function \code{cxxfunction}.  It provides an extension which
works particularly well with \pkg{Rcpp} via so-called `plugins' which provide
information about additional header file and
library locations. 

The use of \pkg{inline} is possible as \pkg{Rcpp} can be installed and
updated just like any other R package using \textsl{e.g.} the
\code{install.packages()} function for initial installation as well as
\code{update.packages()} for upgrades.  So even though R / C++ interfacing
would otherwise require source code, the \pkg{Rcpp} library is always provided
ready for use as a pre-built library through the CRAN package
mechanism.\footnote{This presumes a platform for which pre-built binaries are
  provided. \pkg{Rcpp} is available in binary form for Windows and OS X users from
  CRAN, and as a \code{.deb} package for Debian and Ubuntu users. For other systems, the
  \pkg{Rcpp} library is automatically built from source during installation
  or upgrades.}

The library and header files provided by \pkg{Rcpp} for use by other packages
are installed along with the \pkg{Rcpp} package. The \code{LinkingTo: Rcpp}
directive in the DESCRIPTION file lets R compute the location of the header
file. The \pkg{Rcpp} package provides appropriate information for the \code{-L}
switch needed for linking via the function \code{Rcpp:::LdFlags()} that
provide this information. It can be used by \code{Makevars} files of other
packages, and \pkg{inline} makes use of it internally so that all of this is
done behind the scenes without the need for explicitly setting compiler or
linker options.

The convolution example provided above can be rewritten for use by
\pkg{inline} as shown below.  The function body is provided by the character
variable \code{src}, the function header is defined by the argument
\code{signature}---and we only need to enable \code{plugin="Rcpp"} to obtain a
new function \code{fun} based on the C++ code in \code{src}: 

\begin{example}
> src <- '
+   Rcpp::NumericVector xa(a);
+   Rcpp::NumericVector xb(b);
+   int n_xa = xa.size(), n_xb = xb.size();
+ 
+   Rcpp::NumericVector xab(n_xa + n_xb - 1);
+   for (int i = 0; i < n_xa; i++)
+     for (int j = 0; j < n_xb; j++)
+        xab[i + j] += xa[i] * xb[j];
+   return xab;
+ '
> fun <- cxxfunction( 
+ \ \ \ \	signature(a="numeric", b="numeric"), 
+ \ \ \ \	src, plugin="Rcpp")
> fun( 1:3, 1:4 )
[1]  1  4 10 16 17 12
\end{example}

\section{Using STL algorithms}

The C++ Standard Template Library (STL) offers a variety of generic
algorithms designed to be used on ranges of elements
\citep{plauger_et_al:stlbook}. A range is any sequence of objects that can be
accessed through iterators or pointers.  All \pkg{Rcpp} classes from the new
API representing vectors (including lists) can produce ranges through their
member functions \code{begin()} and \code{end()}, effectively supporting
iterating over elements of an R vector.

The following code illustrates how \pkg{Rcpp} might be used
to emulate a 
simpler\footnote{The version of \code{lapply} does not allow use of the
ellipsis (\code{...}).} version of \code{lapply}
using the \code{transform} algorithm from the STL. 

\begin{example}
> src <- '
+   Rcpp::List input(data); 
+   Rcpp::Function f(fun);
+   Rcpp::List output(input.size());
+   std::transform( 
+   	input.begin(), input.end(), 
+   	output.begin(), 
+   	f );
+   output.names() = input.names();
+   return output;
+   '
> cpp_lapply <- cxxfunction(
+ \ \ signature(data="list", fun = "function"), 
+ \ \ src, plugin="Rcpp")
\end{example}

We can use this to calculate a summary of each 
column of the \code{faithful} dataset included with R.

\begin{example}
> cpp_lapply( faithful, summary )
$eruptions
 Min. 1st Qu.  Median    Mean 3rd Qu.    Max. 
1.600   2.163   4.000   3.488   4.454   5.100 

$waiting
 Min. 1st Qu.  Median    Mean 3rd Qu.    Max. 
 43.0    58.0    76.0    70.9    82.0    96.0 
\end{example}



\section{Error handling}

Code that uses both R and C++ has to deal with two concurrent
error handling models. \pkg{Rcpp} simplifies this and allows both 
systems to work together.

\subsection{C++ exceptions in R}

The internals of the R condition mechanism and the implementation of 
C++ exceptions are both based on a layer above POSIX jumps. These layers 
both assume total control over the call stack and should not be used together
without extra precaution. \pkg{Rcpp} contains facilities to combine both systems
so that C++ exceptions are caught and recycled into the R condition 
mechanism. 

\pkg{Rcpp} defines the \code{BEGIN\_RCPP} and \code{END\_RCPP} macros that should 
be used to bracket code that might throw C++ exceptions. 

\begin{example}
RcppExport SEXP fun( SEXP x )\{
BEGIN_RCPP
    int dx = Rcpp::as<int>(x);
    if( dx > 10 ) 
        throw std::range_error("too big");
    return Rcpp::wrap( dx * dx); 
END_RCPP
\}
\end{example}

The macros are simply defined to avoid code repetition, they expand to 
simple try/catch blocks: 

\begin{example}
RcppExport SEXP fun( SEXP x )\{
    try\{
        int dx = Rcpp::as<int>(x);
        if( dx > 10 ) 
            throw std::range_error("too big");
        return Rcpp::wrap( dx * dx); 
    \} catch( std::exception& __ex__ )\{ 
        forward_exception_to_r( __ex__ );
    \} catch(...)\{ 
        ::Rf_error( "c++ exception "
                    "(unknown reason)" );
    \}
\}
\end{example}

Using \code{BEGIN\_RCPP} and \code{END\_RCPP} --- or the expanded versions ---
guarantess that the stack is first unwound in terms of C++ exceptions, before 
the problem is converted to the standard R error management system (\code{Rf\_error}).

The \code{forward\_exception\_to\_r} function uses run-time type information to 
extract information about the class of the C++ exception and its message, so that 
dedicated handlers can be installed on the R side. 

\begin{example}
> f <- function(x) .Call( "fun", x )
> tryCatch( f( 12 ), 
+    "std::range_error" = function(e) \{
+       conditionMessage( e )
+    \} )
[1] "too big"
> tryCatch( f( 12 ), 
+    "std::range_error" = function(e) \{
+       class( e )
+    \} )
[1] "std::range_error" "C++Error"
[3] "error"            "condition" 
\end{example}

\subsection{R error in C++}

R currently does not offer C-level mechanisms to deal with errors. To 
overcome this problem, \pkg{Rcpp} uses the \code{Rcpp::Evaluator}
class to evaluate an expression with an R-level \code{tryCatch}
block. The error, if any, that occurs while evaluating the 
function is then translated into an C++ exception that can be dealt with using 
regular C++ try/catch syntax.

\section{Performance comparison}

In this section, we present several different ways to leverage \pkg{Rcpp} to 
rewrite the convolution example taken from \cite{R:exts}. 

As part of the redesign of \pkg{Rcpp}, data copy is kept to the
absolute minimum: the \code{RObject} class and all its derived
classes are just a container for a \code{SEXP} object. We let R perform
all memory management and access data though the macros or functions
offered by the standard R API. 

The implementation of the \code{operator[]} is designed to be as 
efficient as possible, using both inlining and caching, 
but even this implementation is still less efficient than the 
reference C implementation described in \cite{R:exts}. 

\pkg{Rcpp} follows design principles from the STL, and classes such 
as \code{NumericVector} expose iterators that can be used for 
sequential scans of the data. Algorithms using iterators are 
usually more efficient than those that operate on objects using the 
\code{operator[]}. The following version illustrate the use of the
\code{NumericVector::iterator}. 

\begin{example}
#include <Rcpp.h>

RcppExport SEXP convolve4cpp(SEXP a, SEXP b)\{
    NumericVector xa(a), xb(b);
    int n_xa = xa.size(), n_xb = xb.size();
    NumericVector xab(n_xa + n_xb - 1);
    
    typedef NumericVector::iterator 
        vec_iterator;
    vec_iterator ia = xa.begin(), 
                 ib = xb.begin();
    vec_iterator iab = xab.begin();
    for (int i = 0; i < n_xa; i++)
        for (int j = 0; j < n_xb; j++) 
            iab[i + j] += ia[i] * ib[j];

    return xab;
\}
\end{example}

One of the focus of recent developments of \pkg{Rcpp} is called `Rcpp sugar', 
and aims to provide R-like syntax in C++. A discussion of Rcpp sugar is 
beyond the scope of this article, but for illustrative purposes we have included
another version of the convolution algorithm based on Rcpp sugar. 

\begin{example}
#include <Rcpp.h>

RcppExport SEXP convolve11cpp(SEXP a, SEXP b) \{
    NumericVector xa(a); int n_xa = xa.size();
    NumericVector xb(b); int n_xb = xb.size();
    NumericVector xab(n_xa + n_xb - 1,0.0);
    
    Range r( 0, n_xb-1 );
    for (int i=0; i<n_xa; i++, r++)
        xab[ r ] += nona(xa[i]) * nona(xb);
    return xab ;
\}
\end{example}

Rcpp sugar allows manipulation of entire subset of vectors at once, thanks to 
the \code{Range} class. Rcpp sugar uses techniques such as expression templates, 
lazy evaluation and loop unrolling to generate very efficient code. 
The \code{nona} template function marks its argument to indicates that it does 
not contain any missing values---an assumption made implicitly by other 
versions---allowing sugar to compute the individual operations without having
to test for missing values. 

\begin{table}[H]
  \begin{center}
    \begin{small}
      \begin{tabular}{lrr}
        \toprule
        Implementation                    & Time in   & Relative \\ 
                                          & millisec  & to R API \\ 
        \cmidrule(r){2-3}
        R API (as benchmark)              &  218       & \\
        Rcpp sugar                        &  145       & 0.67 \\
        \code{NumericVector::iterator}    &  217       & 1.00 \\
        \code{NumericVector::operator[]}  &  282       & 1.29 \\
        \code{RcppVector<double>}         &  683       & 3.13 \\
        \bottomrule
      \end{tabular}
    \end{small}
    \caption{Performance for convolution example}
    \label{tab:benchmark}
  \end{center}
\end{table}

We have benchmarked the various implementations by averaging over 5000 calls 
of each function with \code{a} and \code{b} containing 200 elements
each.\footnote{The code for this example is contained in the directory
  \code{inst/examples/ConvolveBenchmarks} in the \pkg{Rcpp} package.} The timings
are summarized in Table~\ref{tab:benchmark}.

The first implementation, written in C and using the traditional R API
provides our base case. It takes advantage of pointer arithmetics, therefore
does not pay the price of C++'s object encapsulation or operator overloading.
%
The slowest implementation comes from the (deprecated) classic \pkg{Rcpp} API. 
It is clearly behind in terms of efficiency. The difference is mainly 
caused by the many unnecessary copies that the older code 
performs. 

The second-slowest solution uses the more efficient new \pkg{Rcpp} API. While
already orders of magnitude faster than the preceding solution, it
illustrates the price of object encapsulation and of calling an overloaded
\code{operator[]} as opposed to using direct pointer arithmetics as in the
reference ase.

The next implementation uses iterators rather than indexing. Its performance
is indistinguishable from the base case. 
This shows that use of C++ does not necessarily imply any performance penalty.

Finally, the fastest implementation uses Rcpp sugar. It performs
significantly better than the base case: explicit loop unrolling provides
vectorization at the C++ level which is responsible for this speedup.  

\section{Recent development}

The \code{Rcpp} package is in active development. Recent work in the
\pkg{Rcpp} package and in packages such as \pkg{RcppArmadillo}
%\citep*{cran:rcpparmadillo} 
focuses on further improving interoperability between R and C++. 
% should we plug the next article ... to be continued
Two core themes are `Rcpp sugar' as well as `Rcpp modules' both of which are
discussed in their dedicated vignettes in the package.  
% Add citations of vignettes ?
% nah

`Rcpp sugar' brings syntactic
sugar at the C++ level, including optimized binary operators and many 
R functions such as \code{ifelse}, \code{sapply}, \code{any}, ... 
The main technique used in Rcpp sugar is
expression templates pioneered by the Blitz++ library \citep{Blitz}
and since adopted 
by projects such as Armadillo \citep{Armadillo}. 
Access to most of the d/p/q/r-variants of the statistical distribution
functions has also been added, enabling the use of expressions such as 
\code{dnorm(X, m, s)} for a numeric vector $X$ and scalars $m$ and
$s$. Similarly, and continuing Table~\ref{fig:rnormCode},  the R expression
\code{rnorm(10L, sd=100)} can now be written in C++ as \code{rnorm(10, 0,
  100)} where C++ semantics require the second parameter to be used. 

`Rcpp modules' allows programmers to expose C++ functions and classes 
at the R level. This offers access to C++ code from R using even less
interface code than by writing accessor function. Modules are inspired from
the \code{Boost.Python} library 
\citep{Boost:Python} that provides similar functionality for Python. C++ Classes
exposed by Rcpp modules are shadowed by reference classes that have been 
introduced in R 2.12.0. 


\section{Summary}

The \code{Rcpp} package presented here greatly simplifies integration of
compiled C++ code with R.

The class hierarchy allows manipulation of R data structures in C++ 
using member functions and operators directly related to the type
of object being used, therefore reducing the level of expertise
required to master the various functions and macros offered by the
internal R API. The classes assume the entire 
responsibility of garbage collection of objects, relieving the 
programmer from book-keeping operations with the protection stack 
and enabling him/her to focus on the underlying problem. 

Data interchange between R and C++ code---performed by the \code{wrap} and
\code{as} template functions---allows the programmer to write logic in terms
of C++ data structures and facilitates use of modern libraries such as the
standard template library and its containers and algorithms. The
\code{wrap()} and \code{as()} template functions are extensible by design and
can be used either explicitly or implicitly throughout the API.
By using only thin wrappers around \code{SEXP} objects and adopting C++
idioms such as iterators, the footprint of the \code{Rcpp} API 
is very lightweight, and does not induces a significant performance price. 

The \code{Rcpp} API offers opportunities to dramatically reduce the complexity 
of code, which should improve code readability, maintainability and reuse.

\bibliography{EddelbuettelFrancois}

\address{Dirk Eddelbuettel\\
  Debian Project\\
  Chicago, IL\\
  USA}\\
\email{edd@debian.org}

\address{Romain Fran\c{c}ois\\
  Professional R Enthusiast\\
  1 rue du Puits du Temple, 34 000 Montpellier\\
  FRANCE}\\
\email{romain@r-enthusiasts.com}

