
\documentclass[shortnames,article]{jss}
%\VignetteIndexEntry{RcppEigen-intro}
%\VignetteKeywords{linear algebra, template programming, C++, R, Rcpp}
%\VignettePackage{RcppEigen}

\usepackage{booktabs,flafter}

\newcommand{\R}{\proglang{R}\ } % NB forces a space so not good before % fullstop etc.
\newcommand{\Cpp}{\proglang{C++}\ }

\author{Douglas Bates\\U Wisconsin/Madison \And Dirk Eddelbuettel\\Debian Project \And Romain Fran\c{c}ois\\R Enthusiasts}
\title{An Introduction to \pkg{RcppEigen}}

\Plainauthor{Douglas Bates, Dirk Eddelbuettel, Romain Fran\c{c}ois} 
\Plaintitle{An Introduction to RcppEigen}
\Shorttitle{An Introduction to RcppEigen}

\Abstract{
  \noindent
  \noindent
  The \pkg{RcppEigen} package provides access from \proglang{R}
  \citep{R:Main} to the \pkg{Eigen} \citep*{Eigen:Web} \proglang{C++}
  template library for numerical linear algebra. \pkg{Rcpp}
  \citep{JSS:Rcpp,CRAN:Rcpp} classes and specializations of the
  \proglang{C++} templated functions \code{as} and \code{wrap} from
  \pkg{Rcpp} provide the ``glue'' for passing objects from \proglang{R} to
  \proglang{C++} and back.  
}

\Keywords{Linear algebra, template programming, \proglang{R}, \proglang{C++}, \pkg{Rcpp}} %% at least one keyword must be supplied
\Plainkeywords{Linear algebra, template programmig, R, C++, Rcpp} %% without formatting

\Address{
  Douglas Bates \\
  Department of Statistics \\
  University of Wisconsin - Madison \\
  Madison, WI, USA \\
  E-mail: \email{bates@stat.wisc.edu} \\
  URL: \url{http://www.stat.wisc.edu/~bates/}\\

  Dirk Eddelbuettel \\
  Debian Project \\
  River Forest, IL, USA\\
  E-mail: \email{edd@debian.org}\\
  URL: \url{http://dirk.eddelbuettel.com}\\
  
  Romain Fran\c{c}ois\\
  Professional R Enthusiast\\
  1 rue du Puits du Temple, 34 000 Montpellier\\
  FRANCE \\
  E-mail: \email{romain@r-enthusiasts.com}\\
  URL: \url{http://romainfrancois.blog.free.fr}
}

%% need no \usepackage{Sweave.sty}

\newcommand{\rank}{\operatorname{rank}}

%% highlights macros
%% Style definition file generated by highlight 2.7, http://www.andre-simon.de/
% Highlighting theme definition:
\newcommand{\hlstd}[1]{\textcolor[rgb]{0,0,0}{#1}}
\newcommand{\hlnum}[1]{\textcolor[rgb]{0,0,0}{#1}}
\newcommand{\hlopt}[1]{\textcolor[rgb]{0,0,0}{#1}}
\newcommand{\hlesc}[1]{\textcolor[rgb]{0.74,0.55,0.55}{#1}}
%\newcommand{\hlstr}[1]{\textcolor[rgb]{0.74,0.55,0.55}{#1}}
\newcommand{\hlstr}[1]{\textcolor[rgb]{0.90,0.15,0.15}{#1}}
%green: \newcommand{\hlstr}[1]{\textcolor[rgb]{0.13,0.67,0.13}{#1}} % 0.74 -> % 0.90; 0.55 -> 0.25
\newcommand{\hldstr}[1]{\textcolor[rgb]{0.74,0.55,0.55}{#1}}
\newcommand{\hlslc}[1]{\textcolor[rgb]{0.67,0.13,0.13}{\it{#1}}}
\newcommand{\hlcom}[1]{\textcolor[rgb]{0.67,0.13,0.13}{\it{#1}}}
\newcommand{\hldir}[1]{\textcolor[rgb]{0,0,0}{#1}}
\newcommand{\hlsym}[1]{\textcolor[rgb]{0,0,0}{#1}}
\newcommand{\hlline}[1]{\textcolor[rgb]{0.33,0.33,0.33}{#1}}
\newcommand{\hlkwa}[1]{\textcolor[rgb]{0.61,0.13,0.93}{\bf{#1}}}
\newcommand{\hlkwb}[1]{\textcolor[rgb]{0.13,0.54,0.13}{#1}}
\newcommand{\hlkwc}[1]{\textcolor[rgb]{0,0,1}{#1}}
\newcommand{\hlkwd}[1]{\textcolor[rgb]{0,0,0}{#1}}
\definecolor{bgcolor}{rgb}{1,1,1}


% ------------------------------------------------------------------------

\begin{document}

%\SweaveOpts{engine=R,eps=FALSE}

\section{Introduction}
\label{sec:intro}

Linear algebra is an essential building block of statistical computing.
Operations such as matrix decompositions, linear program solvers, and
eigenvalue / eigenvector computations are used in many estimation and
analysis routines. As such, libraries supporting linear algebra have long been
provided by statistical programmers for different programming languages and
environments. \proglang{C++}, one of the central modern languages for numerical
and statistical computing, can be extended particularly well due to its
object-oriented nature, and numerous class libraries providing linear algebra
routines have been written over the years.
% Could cite Eddelbuettel (1996) here, but a real survey would be better.

As both the \proglang{C++} language and standards have evolved
\citep{Meyers:2005:EffectiveC++,Meyers:1995:MoreEffectiveC++}, so have the
compilers implementing the language.  Relatively modern language constructs
such as template meta-programming are particularly useful. It provides both
overloading of operations (allowing expressive code in the compiled language
similar to what can be done in scripting languages) and can shift some of the
computational burden from the run-time to the compile-time (though a more
detailed discussions of template programming is however beyond this
paper). \cite{Veldhuizen:1998:Blitz} provided an early and influential
implementation that already demonstrated key features of this approach.  Its
usage however was held back at the time by the somewhat limited availability
of compilers implementing all necessary features of the \proglang{C++}
language.

This situation has greatly improved over the last decade, and many more such
libraries have been contributed. One such \proglang{C++} library is
\pkg{Eigen} by \citet*{Eigen:Web}. \pkg{Eigen} started as a sub-project to
KDE (a popular Linux desktop environment), initially focussing on fixed-size
matrices which are projections in a visualization application. \pkg{Eigen}
grew from there and has over the course of about a decade produced three
major releases with ``Eigen3'' being the current version.

\pkg{Eigen} is of interest as the \proglang{R} system for statistical
computation and graphics \citep{R:Main} is itself easily extensible. This is
particular true via the \proglang{C} language that most of \proglang{R}'s
compiled core parts are written in, but also for the \proglang{C++} language
which can interface with \proglang{C}-based systems rather easily. The manual
``Writing R Extensions'' \citep{R:Extensions} is the basic reference for
extending \proglang{R} with either \proglang{C} or \proglang{C++}.

The \pkg{Rcpp} package by \citet{JSS:Rcpp,CRAN:Rcpp} facilitates extending
\proglang{R} with \proglang{C++} code by providing seamless object mapping
between both languages.
%
As stated in the \pkg{Rcpp} \citep{CRAN:Rcpp} vignette, ``Extending \pkg{Rcpp}''
\begin{quote}
  \pkg{Rcpp} facilitates data interchange between \proglang{R} and
  \proglang{C++} through the templated functions \texttt{Rcpp::as} (for
  conversion of objects from \proglang{R} to \proglang{C++}) and
  \texttt{Rcpp::wrap} (for conversion from \proglang{C++} to \proglang{R}).
\end{quote}
The \pkg{RcppEigen} package provides the header files composing the
\pkg{Eigen} \proglang{C++} template library and implementations of
\texttt{Rcpp::as} and \texttt{Rcpp::wrap} for the \proglang{C++}
classes defined in \pkg{Eigen}.

The \pkg{Eigen} classes themselves provide high-performance,
versatile and comprehensive representations of dense and sparse
matrices and vectors, as well as decompositions and other functions
to be applied to these objects.  The next section introduces some
of these classes and shows how to interface to them from \proglang{R}.

\section{Eigen classes}
\label{sec:eclasses}

\pkg{Eigen} \citep*{Eigen:Web} is a \proglang{C++} template
library providing classes for many forms of matrices, vectors, arrays
and decompositions.  These classes are flexible and comprehensive
allowing for both high performance and well structured code
representing high-level operations. \proglang{C++} code based on Eigen
is often more like \proglang{R} code, working on the ``whole object'',
rather than compiled code in other languages where operations often must be
coded in loops.

As in many \proglang{C++} template libraries using template meta-programming
\citep{Abrahams+Gurtovoy:2004:TemplateMetaprogramming}, the templates
themselves can be very complicated.  However, \pkg{Eigen} provides
\code{typedef}s for common classes that correspond to \proglang{R} matrices and
vectors, as shown in Table~\ref{tab:REigen}, and this paper will use these
\code{typedef}s throughout this document.
\begin{table}[tb]
  \caption{Correspondence between R matrix and vector types and classes in the \code{Eigen} namespace.}
  \label{tab:REigen}
  \centering
  \begin{tabular}{l l}
    \toprule
    \multicolumn{1}{c}{\proglang{R} object type} & \multicolumn{1}{c}{\pkg{Eigen} class typedef}\\
    \midrule
    numeric matrix                          & \code{MatrixXd}\\
    integer matrix                          & \code{MatrixXi}\\
    complex matrix                          & \code{MatrixXcd}\\
    numeric vector                          & \code{VectorXd}\\
    integer vector                          & \code{VectorXi}\\
    complex vector                          & \code{VectorXcd}\\
    \code{Matrix::dgCMatrix} \phantom{XXX}  & \code{SparseMatrix<double>}\\
    \bottomrule
  \end{tabular}
\end{table}

The \proglang{C++} classes shown in Table~\ref{tab:REigen} are in the
\code{Eigen} namespace, which means that they must be written as
\code{Eigen::MatrixXd}.  However, if one prefaces the use of these class
names with a declaration like

%% Alternatively, use 'highlight --enclose-pre --no-doc --latex --style=emacs --syntax=C++'
%% as the command invoked from C-u M-|
%% For version 3.5 of highlight this should be
%%  highlight --enclose-pre --no-doc --out-format=latex --syntax=C++
%%
%% keep one copy to redo later
%%
%% using Eigen::MatrixXd;
%%
\begin{quote}
  \noindent
  \ttfamily
  \hlstd{}\hlkwa{using\ }\hlstd{Eigen}\hlopt{::}\hlstd{MatrixXd}\hlopt{;}\hlstd{}\hspace*{\fill}\\
  \mbox{}
  \normalfont
  \normalsize
\end{quote}
then one can use these names without the namespace qualifier.

\subsection{Mapped matrices in Eigen}
\label{sec:mapped}

Storage for the contents of matrices from the classes shown in
Table~\ref{tab:REigen} is allocated and controlled by the class
constructors and destructors.  Creating an instance of such a class
from an \proglang{R} object involves copying its contents.  An
alternative is to have the contents of the \proglang{R} matrix or
vector mapped to the contents of the object from the Eigen class.  For
dense matrices one can use the Eigen templated class \code{Map}, and for
sparse matrices one can deploy the Eigen templated class \code{MappedSparseMatrix}.

One must, of course, be careful not to modify the contents of the
\proglang{R} object in the \proglang{C++} code.  A recommended
practice is always to declare mapped objects as {\ttfamily\hlkwb{const}\normalfont}.

\subsection{Arrays in Eigen}
\label{sec:arrays}

For matrix and vector classes \pkg{Eigen} overloads the \texttt{`*'}
operator to indicate matrix multiplication.  Occasionally
component-wise operations instead of matrix operations are preferred.  The
\code{Array} templated classes are used in \pkg{Eigen} for
component-wise operations.  Most often the \code{array()} method is used
for Matrix or Vector objects to create the array.  On those occasions
when one wishes to convert an array to a matrix or vector object
the \code{matrix()} method is used.

\subsection{Structured matrices in Eigen}
\label{sec:structured}

There are \pkg{Eigen} classes for matrices with special structure such
as symmetric matrices, triangular matrices and banded matrices.  For
dense matrices, these special structures are described as ``views'',
meaning that the full dense matrix is stored but only part of the
matrix is used in operations.  For a symmetric matrix one needs to
specify whether the lower triangle or the upper triangle is to be used as
the contents, with the other triangle defined by the implicit symmetry.

\section{Some simple examples}
\label{sec:simple}

\proglang{C++} functions to perform simple operations on matrices or
vectors can follow a pattern of:
\begin{enumerate}
\item Map the \proglang{R} objects passed as arguments into Eigen objects.
\item Create the result.
\item Return \code{Rcpp::wrap} applied to the result.
\end{enumerate}

An idiom for the first step is
%\begin{lstlisting}[language=C++]
% using Eigen::Map;
% using Eigen::MatrixXd;
% using Rcpp::as;

% const Map<MatrixXd>  A(as<Map<MatrixXd> >(AA));
%\end{lstlisting}
\begin{quote}
  \noindent
  \ttfamily
  \hlstd{}\hlkwa{using\ }\hlstd{Eigen}\hlsym{::}\hlstd{Map}\hlsym{;}\hspace*{\fill}\\
  \hlstd{}\hlkwa{using\ }\hlstd{Eigen}\hlsym{::}\hlstd{MatrixXd}\hlsym{;}\hspace*{\fill}\\
  \hlstd{}\hlkwa{using\ }\hlstd{Rcpp}\hlsym{::}\hlstd{as}\hlsym{;}\hspace*{\fill}\\
  \hlstd{}\hspace*{\fill}\\
  \hlkwb{const\ }\hlstd{Map}\hlsym{$<$}\hlstd{MatrixXd}\hlsym{$>$}\hlstd{\ \ }\hlsym{}\hlstd{}\hlkwd{A}\hlstd{}\hlsym{(}\hlstd{as}\hlsym{$<$}\hlstd{Map}\hlsym{$<$}\hlstd{MatrixXd}\hlsym{$>$\ $>$(}\hlstd{AA}\hlsym{));}\hlstd{}\hspace*{\fill}\\
  \mbox{}
  \normalfont
\end{quote}
where \code{AA} is the name of the R object (called an \code{SEXP} in
\proglang{C} and \proglang{C++}) passed to the \proglang{C++} function.

The \code{cxxfunction} from the \pkg{inline} package \citep*{CRAN:inline} for
\proglang{R} and its \pkg{RcppEigen} plugin provide a convenient way of
developing and debugging the \proglang{C++} code.  For actual production code
one generally incorporates the \proglang{C++} source code files in a package
and include the line \code{LinkingTo: Rcpp, RcppEigen} in the package's
\code{DESCRIPTION} file.  The \code{RcppEigen.package.skeleton} function
provides a quick way of generating the skeleton of a package using
\pkg{RcppEigen} facilities.

The \code{cxxfunction} with the \code{"Rcpp"} or \code{"RcppEigen"}
plugins has the \code{as} and \code{wrap} functions already defined as
\code{Rcpp::as} and \code{Rcpp::wrap}.  In the examples below
these declarations are omitted.  It is important to remember that they are
needed in actual \proglang{C++} source code for a package.

The first few examples are simply for illustration as the operations
shown could be more effectively performed directly in \proglang{R}.
Finally, the results from \pkg{Eigen} are compared to those from the direct
\proglang{R} results.

\subsection{Transpose of an integer matrix}
\label{sec:transpose}

The next \proglang{R} code snippet creates a simple matrix of integers
\begin{verbatim}
> (A <- matrix(1:6, ncol=2))
     [,1] [,2]
[1,]    1    4
[2,]    2    5
[3,]    3    6
> str(A)
 int [1:3, 1:2] 1 2 3 4 5 6
> 
\end{verbatim}
and, in Figure~\ref{trans}, the \code{transpose()} method for the
\code{Eigen::MatrixXi} class is used to return the transpose of the supplied matrix. The \proglang{R}
matrix in the \code{SEXP} \code{AA} is mapped to an
\code{Eigen::MatrixXi} object then the matrix \code{At} is constructed
from its transpose and returned to \proglang{R}.

\begin{figure}[htb]
  \begin{quote}
    \noindent
    \ttfamily
    \hlstd{}\hlkwa{using\ }\hlstd{Eigen}\hlsym{::}\hlstd{Map}\hlsym{;}\hspace*{\fill}\\
    \hlstd{}\hlkwa{using\ }\hlstd{Eigen}\hlsym{::}\hlstd{MatrixXi}\hlsym{;}\hspace*{\fill}\\
    \hlstd{}\hlstd{\ \ \ \ \ \ \ \ \ \ \ \ \ \ \ \ \ }\hlstd{}\hlslc{//\ Map\ the\ integer\ matrix\ AA\ from\ R}\hspace*{\fill}\\
    \hlstd{}\hlkwb{const\ }\hlstd{Map}\hlsym{$<$}\hlstd{MatrixXi}\hlsym{$>$}\hlstd{\ \ }\hlsym{}\hlstd{}\hlkwd{A}\hlstd{}\hlsym{(}\hlstd{as}\hlsym{$<$}\hlstd{Map}\hlsym{$<$}\hlstd{MatrixXi}\hlsym{$>$\ $>$(}\hlstd{AA}\hlsym{));}\hspace*{\fill}\\
    \hlstd{}\hlstd{\ \ \ \ \ \ \ \ \ \ \ \ \ \ \ \ \ }\hlstd{}\hlslc{//\ evaluate\ and\ return\ the\ transpose\ of\ A}\hspace*{\fill}\\
    \hlstd{}\hlkwb{const\ }\hlstd{MatrixXi}\hlstd{\ \ \ \ \ \ }\hlstd{}\hlkwd{At}\hlstd{}\hlsym{(}\hlstd{A}\hlsym{.}\hlstd{}\hlkwd{transpose}\hlstd{}\hlsym{());}\hspace*{\fill}\\
    \hlstd{}\hlkwa{return\ }\hlstd{}\hlkwd{wrap}\hlstd{}\hlsym{(}\hlstd{At}\hlsym{);}\hlstd{}\hspace*{\fill}
    \normalfont
  \end{quote}
  \caption{\textbf{transCpp}: Transpose a matrix of integers}
  \label{trans}
\end{figure}

The next \proglang{R} snippet compiles and links the \proglang{C++} code
segment (stored as text in a variable named \code{transCpp}) into an
executable function \code{ftrans} and then checks that it works as intended
by compariing the output to an explicit transpose of the matrix argument.
\begin{verbatim}
> ftrans <- cxxfunction(signature(AA="matrix"), transCpp, plugin="RcppEigen")
> (At <- ftrans(A))
     [,1] [,2] [,3]
[1,]    1    2    3
[2,]    4    5    6
> stopifnot(all.equal(At, t(A)))
\end{verbatim}


For numeric or integer matrices the \code{adjoint()} method is
equivalent to the \code{transpose()} method.  For complex matrices, the
adjoint is the conjugate of the transpose.  In keeping with the
conventions in the \pkg{Eigen} documentation, in what follows,
the \code{adjoint()} method is used to create the transpose of numeric or
integer matrices.




\bibliography{Rcpp}

\end{document}

%%% Local Variables: 
%%% mode: latex
%%% TeX-master: t
%%% End: 



